%!TEX root = ../Electrodynamics.tex

\subsection{Гармонические волны в линиях передачи. Выражение для векторного потенциала. Дифференциальное уравнение для
поперечной волновой функции $\psi(\vec{r_{\perp}})$. Понятия продольного и поперечного волнового числа. Выражения для
полей ТЕ, ТМ, ТЕМ. Импедансная связь между поперечными компонентаит электрического и магнитного полей и понятие
поперечного волнового сопротивления}

Линия передач - это любая цилиндрическая система. В них различают продольное $z$ и поперечное $\vec{r_{\perp}} = r_{\perp}(r,\theta)$ направление. При описании
таких систем проще использовать векторный потенциал $\vec{A}$, который должен удовлетворять уравнению Гельмгольца (для амплитуд):
\begin{align*}
  &\Delta \vec{A^e} + k^2 \vec{A^e} = -\frac{4 \pi \mu}{c} \vec{j^e}  =0 \\
  &\vec{B} = rot \vec{A^e}
  \label{eq:1}
\end{align*}
0 потому что случай, где нет сторонних источников. Запишем поля в ЛП, когда волна бежит вдоль оси $Oz$:
\begin{equation}
  \vec{E}(\vec{r}_{\perp},z,\theta) = \vec{E_0}(\vec{r}_{\perp}) e^{i(wt-hz)},
  \label{eq:2}
\end{equation}
где h - \textbf{продольное волновое число} (постоянная распространения). Реальное поля в таком случае записывается как:
\begin{equation}
  E_{R_x} = \Re{E_x} = |E_x(\vec{r}_{\perp})|\cos (wt-hz + \varphi(\vec{r}_{\perp}))
  \label{eq:3}
\end{equation}
Запишем веторный потенциал в следующем виде:
\begin{equation}
  \vec{A^e} = \psi^e(\vec{r}_{\perp})e^{-ihz}\vec{z_0},
  \label{eq:4}
\end{equation}
где $\psi^e(\vec{r}_{\perp})$ - \textbf{поперечная волновая функция}. Запишем теперь поля $\vec{E}$ и $\vec{H}$ через
$\psi^e(\vec{r}_{\perp})$. Вспомним выражение полей через векторный потенциал:
\begin{align*}
  &\vec{H} = \frac{1}{\mu}rot \vec{A^e}\\
  &\vec{E} = -\nabla \varphi - \frac{1}{c}\frac{\partial\vec{A^e}}{\partial t} = \frac{1}{ik_0\varepsilon\mu}(\nabla div + k^2)\vec{A^e},
  \label{eq:5}
\end{align*} 
где $k = \frac{w}{c}\sqrt{\varepsilon\mu}, k_0 = \frac{w}{c}$. При подстановке выражения для $\vec{A^e}$, для компонент
векторов в случае \textbf{TM} - волны получим(надо расписать такие вещи как $\Div{\vec{A^e}},~ \nabla \Div{\vec{A^e}},~\Rot{\vec{A^e}}$):
\begin{align*}
  &E_z = \frac{\varkappa^2}{ik_0\varepsilon\mu} \psi^e(\vec{r}_{\perp})\cdot e^{i(wt-hz)}\\
  &\vec{E_{\perp}} = -\frac{h}{k_0\varepsilon\mu}\nabla_{\perp} \psi^e(\vec{r}_{\perp})\cdot e^{i(wt-hz)}\\
  &\vec{H_{\perp}} = \frac{1}{\mu}[\nabla_{\perp} \psi^e(\vec{r}_{\perp})\times\vec{z_0}]\cdot e^{i(wt-hz)}\\
  &H_z = 0
\end{align*}
\textbf{ТМ-волна} - поперечная магнитная волна (Магнитное поле имеет только поперечную компоненту. Поле $\vec{E}$ имеет и поперечное и
продольное направление). 

Потенциал $\vec{A^e}$, при любой зависимости от времени, должен удовлетворять волновому уравнению:
\begin{equation}
  \Delta \vec{A^e} - \frac{\varepsilon\mu}{c^2} \frac{\partial^2\vec{A^e}}{\partial t^2} = 0
\end{equation}
В нашем случае, когда векторный потенциал имеет вид $\vec{A^e} = \psi^e(\vec{r}_{\perp})e^{-ihz}\vec{z_0}$, для
гармонических полей справедливы следующие переходы:
\begin{equation}
  \frac{\partial}{\partial t} \Rightarrow iw,~ \Delta \vec{A^e} + k^2 \vec{A^e} =0,~ k^2 = \frac{w^2}{c^2}\varepsilon\mu
\end{equation}
Рассмотри для $z$-компоненты:
\begin{equation}
  \Delta A^e_z + k^2 A^e_z =0,~ \Delta = \Delta_{\perp} + \frac{\partial^2}{\partial z^2}
\end{equation}
\begin{equation}
  \frac{\partial^2}{\partial z^2} \Rightarrow -h^2,~ \text{т.к.} A^e_z = \psi^e(\vec{r}_{\perp})e^{-ihz}
\end{equation}
\begin{equation}
  \Delta_{\perp}\psi^e+\underbrace{(k^2-h^2)}_{\varkappa^2}\psi^e=0
\end{equation}
\begin{equation}
  \Delta_{\perp}\psi^e+\varkappa^2\psi^e=0
\end{equation}
$\varkappa^2$ - \textbf{поперечное волновое число}. Если поле удовлетворяет уравнению выше, то такое поле удоветворяет
уравнениям Максвелла.

Аналогично сделаем для \textbf{ТЕ} - волны.

\textbf{ТЕ-волна} - поперечная электрическая волна (Электрическое поле имеет только поперечную компоненту. Магнитное поле имеет и поперечное и
продольное направление). По принципу двойственности производим замены:
\begin{equation}
  \vec{E}\rightarrow\vec{H},~ \vec{H}\rightarrow-\vec{E},~ \varepsilon \leftrightarrow \mu 
\end{equation}
\begin{align*}
  &H_z = \frac{\varkappa^2}{ik_0\varepsilon\mu} \psi^m(\vec{r}_{\perp})\cdot e^{i(wt-hz)}\\
  &\vec{H_{\perp}} = -\frac{h}{k_0\varepsilon\mu}\nabla_{\perp} \psi^m(\vec{r}_{\perp})\cdot e^{i(wt-hz)}\\
  &\vec{E_{\perp}} = -\frac{1}{\mu}[\nabla_{\perp} \psi^m(\vec{r}_{\perp})\times\vec{z_0}]\cdot e^{i(wt-hz)}\\
  &E_z = 0
\end{align*}
Вообще говоря, $\psi^e$ и $\psi^m$ могут быть различными, поэтому выше вместо $\psi^e$ записано $\psi^m$ . Аналогично
для $\psi^m$ требуется выполнение:
\begin{equation}
  \Delta_{\perp}\psi^m+\varkappa^2\psi^m=0
\end{equation}
ТЕ, ТМ волны - это решения уравнений Максвелла. однак может быть еще один тип решений - \textbf{ТЕМ} - волны.
Рассмотрим случай $\varkappa = 0,~ h=k$:
\begin{align*}
  &H_z = E_z = 0 \\
  &\vec{E_{\perp}} = -\frac{1}{\sqrt{\varepsilon\mu}}\nabla_{\perp} \psi\cdot e^{i(wt-kz)}\\
  &\vec{H_{\perp}} = \frac{1}{\mu}[\nabla_{\perp} \psi\times\vec{z_0}]\cdot e^{i(wt-kz)}\\
  &\Delta_{\perp}\psi=0
\end{align*}
\textbf{ТЕM-волна} - чисто поперечная волна (Электрическое поле имеет только поперечную компоненту, как и магнитное).

Что имеем в итоге:
\begin{itemize}
  \item Поля выражаются через поперечную волновую функцию
  \item Продольные компоненты полей пропорциональны $\psi$
  \item Поперечные компоненты полей пропорциональны $\nabla_{\perp} \psi$
\end{itemize}
Т.е. если заданы $\psi^e,\psi^m$, то можно полностью найти поля. Из формул также видно следующее соотношение:
\begin{align*}
  \vec{E}_{\perp} = \eta_{\perp \text{в}}[\vec{H}_{\perp}\times \vec{z_0}] ,
\end{align*}
где $\eta_{\perp \text{в}}$  - \textbf{поперечное волновое сопротивление} - отношение между поперечными компонентами полей в
бегущей волне $\eta_{\perp \text{в}} = \frac{E_{\perp}}{H_{\perp}}$.
Для различных типов волн записывается как:
\begin{align*}
  \text{ТЕ(+),ТМ(-) - волны: }&\eta_{\perp \text{в}} = \sqrt{\frac{\mu}{\varepsilon}}\left(\frac{k}{h}\right)^{\pm1}\\
  \text{ТЕМ - волны: }&\eta_{\perp \text{в}} = \sqrt{\frac{\mu}{\varepsilon}}
\end{align*}
Заметим, что в бегущей волне поля зависят от координат, а их отношение - $\eta_{\perp \text{в}}$ - нет. В стоячей волне
это не так.

