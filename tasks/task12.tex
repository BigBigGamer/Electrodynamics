%!TEX root = ../Electrodynamics.tex
\subsection{Общая постановка задачи о собственных электромагнитных колебаниях в полых резонаторах с идеально проводящими
стенками. Действительность собственных частот идеального резонатора.}

\begin{figure}[h!]
    \centering
    \includegraphics[width = .3\linewidth]{example-image-c}
    \caption{}
\end{figure}
Рассматриваем любую металлическую полость произвольной формы. Для идеального проводника выполняются граничные условия на
поверхности проводника $S$:
\begin{equation}
    E_{\tau}\big|_S = 0,~ H_n\big|_S = 0    
\end{equation}
Для собственных колебаний необходимо решить систему из уравнений Максвелла:
\begin{equation}
    \left\{
    \begin{aligned}
        &\Rot{\vec{H}} = i\frac{w}{c}\epsilon \vec{E}\\
        &\Rot{\vec{E}} = -i\frac{w}{c}\mu \vec{H}
    \end{aligned}\right.    
\end{equation}
Возьмем ротор от второго уравнения, и подставим $\Rot{\vec{H}}$ из первого, получим:
\begin{equation}
    -\Rot{\Rot{\vec{E}}}+k^2\vec{E} = 0,~ k = \frac{w}{c}\sqrt{\mu\epsilon}    
\end{equation}
Необходмо решить
\begin{equation}
    \Delta \vec{E} +k^2 \vec{E} = 0, \text{ при условии}
    \left\{
    \begin{aligned}
        &\Div{E} = 0\\
        &E_{\tau}\big|_S = 0
    \end{aligned}\right.    
\end{equation}