%!TEX root = ../Electrodynamics.tex

\documentclass[a4paper,14pt]{extarticle}

\usepackage{cmap}
\usepackage[T2A]{fontenc}
\usepackage[utf8x]{inputenc}
% \usepackage{mathptmx}
\usepackage[english, russian]{babel}

\usepackage{misccorr}
\usepackage{amssymb,amsfonts,amsmath,amsthm}  
\usepackage{indentfirst}
\usepackage[usenames,dvipsnames]{color} 
\usepackage[unicode,hidelinks]{hyperref}
\usepackage{makecell,multirow} 
\usepackage{ulem}
\usepackage{graphicx,wrapfig}
\graphicspath{{img/}}

\renewcommand{\labelenumii}{\theenumii)} 
\newcommand{\mean}[1]{\langle#1\rangle}

\DeclareMathOperator{\Div}{div}
\DeclareMathOperator{\const}{const}
%%%%%%%%%%%%%%%%%%%%%%%%%%%%%%%%%%%%%%%%%%%%%%%%%%%%%%%%%%%%%%%%%%%%%%%%%%%%%%%
%%%%%%%%%%%%%%%%%%%%%%%%%%%%%%%%%%%%%%%%%%%%%%%%%%%%%%%%%%%%%%%%%%%%%%%%%%%%%%%
\usepackage{float}
\usepackage[mode=buildnew]{standalone}
\usepackage[outline]{contour}
\usepackage{tocloft}
\renewcommand{\cftsecleader}{\cftdotfill{\cftdotsep}} % for parts
% \renewcommand{\cftchapleader}{\cftdotfill{\cftdotsep}} % for chapters
\usepackage{pgfplots,pgfplotstable,booktabs,colortbl}
\usepackage{physics}
\usepackage{mathtools}
\mathtoolsset{showonlyrefs=true}

% \newcommand*\dotvec[1][1,1]{\crossproducttemp#1\relax}
% \def\crossproducttemp#1,#2\relax{{\qty[\vec{#1}\times\vec{#2}\,]}}

% \newcommand*\prodvec[1][1,1]{\crossproducttempa#1\relax}
% \def\crossproducttempa#1,#2\relax{{\qty[{#1}\times{#2}\,]}}
% \usepackage{showframe}



\usepackage[]{geometry}
\geometry		
	{
		left			=	2cm,
		right 			=	2cm,
		top 			=	2cm,
		bottom 			=	2cm,
		bindingoffset	=	0cm
  }
  

	%применим колонтитул к стилю страницы
% \pagestyle{fancy} 
% 	%очистим "шапку" страницы
% \fancyhead{} 
% 	%очистим "подвал" страницы
% \fancyfoot{} 
% 	% номер страницы в нижнем колинтуле в центре
% \fancyfoot[C]{\thepage} 

\linespread{1} 
\setlength{\parindent}{1.25cm}
\frenchspacing 
\usepackage{setspace}
\setlength{\tabcolsep}{20pt}
\renewcommand{\arraystretch}{1.5}
\usepackage{xcolor}

